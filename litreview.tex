\chapter{Literature Review}
\section{Ascidian tail development}

The notochord is one of the distinguishing characteristics of chordates: it is responsible for the extension of the larval tail which is typical for chordate body plan. In their adult forms, ascidians and their vertebrate cousins bear little resemblance to one another, however during development they have similar body plans that include the notochord \cite{jeffery_minireview_2002}. Ascidians are known for their bilateral and invariant cell cleavage, and their development is well described up to the gastrulation stage \cite{nishida_cell_1983,nishida_cell_1985,nishida_cell_1987}. Like vertebrates such as Xenopus, ascidians depend on maternally localized determinants to regulate cell moments and division, but while the development of the early body plans are similar the location and identity of these determinants are different \cite{lemaire_ascidians_2008}. Solitary ascidian notochords typically originate from two cell lineages, with the primary notochord deriving from the ``A'' blastomere and the secondary notochord deriving from the ``B'' blastomere \cite{nishida_cell_1983}.  Both ``A'' and ``B'' blastomeres can be identified at the 4-cell embryonic stage. At this stage the blastomeres are labeled according to the Conklin convention: ``a'' and ``A'' label the anterior animal and vegetal blastomeres, while ``b'' and ``B'' label the posterior animal and vegetal blastomeres \cite{conklin_organization_1905} . Although the notochords cells have been traced back to the 4-cell stage, notochord induction does not occur until the 32-cell stage. By the 64-cell stage there are 10 notochord cell precursors, the 8 primary precursor notochord cells\textemdash A lineage\textemdash which are no longer multipotent, and the 2 secondary notochord cells which are not restricted until the 110-cell stage \cite{nishida_cell_1985,yasuo_ascidian_1994,yasuo_conservation_1998,lemaire_unfolding_2009}. Two additional stages of cell division occur, one at gastrulation and one at neurulation, ending with the 40 notochord cells that are typical of most solitary ascidian tadpole larvae \cite{conklin_organization_1905}. At the onset of neurulation the notochord begins to form, a process that includes the closing of the neural tube and posterior movement of the notochord and muscle cells, followed by the mediolateral convergence of the notochord cells to the midline and then the polarization and intercalation of the cells through a process known as convergence and extension\cite{swalla_mechanisms_1993}. At this point the larval tail is constructed of a notochord flanked by 3 rows of muscles on each side, and both notochord and muscle cell derive from the same blastomeres \cite{nishida_cell_1985}. While the arrangement of the notochord cells is a stochastic process, the anterior 32-cells\textemdash primary notochord cells\textemdash are always formed by the A7.3 and A7.7 blastomere and the posterior most 8\textemdash secondary\textemdash notochord cells are always formed by the B8.6 blastomere; however, the ordering of the 32 most anterior is not determinate, in that cells from both the A7.3 and A7.7 intercalate in a random order (Figure~\ref{fig:noto_cells})\cite{nishida_cell_1983,nishida_cell_1985,miyamoto_formation_1985, swalla_mechanisms_1993,kourakis_one-dimensional_2014}. This process, along with muscle cell development, drives the formation of the larval tail \cite{miyamoto_formation_1985, jeffery_factors_1992,swalla_mechanisms_1993}.
\begin{figure}[tbp]
\centering
\includegraphics[scale=0.5]{figures/phylogeny.pdf}
\caption{\textbf{Phylographic sketch.} Phylogenetic placement of Cnidaria and the Bilateria (Protostomata and Deuterostoma). Deuterostomes, which in Greek means second mouth, are distinguished in the Bilateria by forming their anus first. Tunicata and Cephalochordata are the sister groups to Vertebrata. This subphyla shares several characteristics; a notochord, a hollow dorsal nerve cord, and a post-anal tail at some point in their life cycles. (Blue) branches are Deuterostomes. For interpretation of the references to color in this and all other figures, the reader is referred to the electronic version of this dissertation.}
\label{fig:phylog}
\end{figure}

\begin{figure}[thbp]
\centering
\includegraphics[scale=0.5]{figures/noto_cells.pdf}
\caption{\textbf{Notochord cells.}The primary notochord cells (red), also known as the A-lineage, are specified at the 64-cell stage. There are a total of 32 primary notochord cells that come from the A7.3 and A7.7 blastomere, and the intercalation of the cells happen in a stochastic manner. The secondary notochord cells (blue) derive from the B8.6 blastomere and are specified at the 110-cell stage, one cell division after the primary notochord cells.}
\label{fig:noto_cells}
\end{figure}

The ascidian tail is used for dispersal during the free-swimming larval stage, where the ascidian locates a substrate to attach. After becoming sessile the ascidian tail is absorbed into the trunk region, undergoing metamorphosis and filter feeding for the duration of their adult life. Although a tailed larvae is typical of most ascidians, several species within the Stolidobranchia order have individually undergone tail-loss, and many of these species fall in the family Molgulidae \cite{berrill_studies_1931, jeffery_evolution_1999, huber_evolution_2000, maliska_molgula_2010}. Species without tails tend to have lower speciation rates and smaller geographical ranges \cite{maliska_developmental_2013}. The tail-less\textemdash anural\textemdash species develop in a similar manner and are indistinguishable from their tailed\textemdash urodele\textemdash counterparts up to late gastrulation \cite{berrill_studies_1931, swalla_interspecific_1990, jeffery_factors_1992}. This was not the case when studying the indirect and direct developing sea urchins, \textit{Heliocidaris tuberculata} and \textit{Heliocidaris erythrogramma}. \textit{H. erythrogramma} for goes the typical pluteus larval body plan and in doing so changes its developmental plan \cite{henry_evolutionary_1990}. After the 4 cell division an unequal three-tiered cleavage is established in \textit{H. tuberculata} and other indirect developers along with the lost of cell division synchrony. In contrast, in \textit{H. erythrogramma} and other direct developers, blastomeres are equivalently sized, cleavage orientation remains parallel and cell division is synchronous. Cell fates also differs between direct and indirect sea urchin species \cite{wray_evolutionary_1989}. The change in body plan in Molgula is a much more recent evolutionary occurrence and gives insight on to early onsite of changing body plans, alternative methods outside of change of cell fate and the ancestral chordate body plan.  

Anural ascidians lack several urodele features including an intercalated and extended notochord, differentiated muscle cells and the otolith sensory organ. The absence of differentiated muscles cells and intercalated notochord are the likely cause of tail-lessness in these species \cite{miyamoto_formation_1985, swalla_interspecific_1990}. The development of several tail-less species has been studied in some detail. \textit{Molgula tectiformis} notochord cells do not divide again after the 10 precursor cells are formed and \textit{M. occulta} stops dividing after 20 cells \cite{jeffery_evolution_1999}. The same occurs in \textit{M. bleizi}, however after the 20 notochord cells are formed, the embryo attempts to make a tail but never completes the process \cite{swalla_novel_1993}. It has also been shown that chordate embryos without fully developed notochord and/or muscle cells do not fully elongate or fail completely to develop a tail \cite{jeffery_evolution_1999,takada_brachyury_2002,stemple_structure_2005}. 
Vertebrates, cephalochordates and most ascidians have tailed larvae and interspecific \textit{Molgula occulta/oculata} hybrids can restore the urodele features, this is evidence that the ancestral chordate had a tailed larvae and that the mechanism for tail development was present in anural ascidians but was lost over evolutionary time \cite{berrill_studies_1931,jeffery_factors_1992}.

In order to study specific mechanisms of tail loss, we can study closely related anural and urodele species.  One such pair of species, \textit{M. oculata} and \textit{M. occulta}, both of the Roscovita clade, have been shown to produce hybrids in lab conditions. Of the known \textit{Molgula} species \textit{M. occulta} and \textit{M. oculata} are the only two that can hybridize. Although \textit{M. occulta} and \textit{M. oculata} have been found to dwell in the same habitat, hybrids have not been found in nature and have only been produced in lab conditions. Fertilizing \textit{M. oculata} eggs with \textit{M. occulta} sperm in most cases produce embryos with fully formed tails. The reciprocal cross (\textit{M. occulta} eggs X \textit{M. oculata} sperm) produces a hybrid embryo with 20 notochord cells like \textit{M. occulta}, however the notochord cells converge and extend like \textit{M. oculata} \cite{swalla_interspecific_1990}. The ascidian tail has been shown to form in the presence of notochord and the absence of muscles cells \cite{miyamoto_formation_1985}. Tail development is similar in short-tailed hybrids, the notochord is not flanked by muscle as in tailed species and the tail is only as long as the notochord \cite{swalla_novel_1993}. Hybrid embryos that develop urodele features are batch specific, and tails develop only in batches of \textit{M. occulta} eggs that express the \textit{p58} protein which is associated with cytoskeleton \cite{swalla_identification_1991,jeffery_factors_1992}. Additionally, in hybrid embryos in which urodele features are restored, the number of cells that express acetylcholinesterase (AChE) in a vestigial muscle cell lineage increased in comparison to hybrids lacking urodele features and \textit{M. occulta} \cite{jeffery_evolutionary_1991}. This along with evidence that the ancestral notochord\textemdash the axochord\textemdash is muscle based \cite{lauri_development_2014}, suggests the need for both notochord and muscles cell lineages for the formation of the ascidian tail. 

\section{Notochord development as seen through \textit{Brachyury}}

\textit{Brachyury}, a T-box transcription factor, has been identified as essential for notochord development \cite{yasuo_conservation_1998}. \textit{Bra} was first discovered in mouse, where heterozygotes develop with shorten tails, and homozygotes fail to form a tail and die in utero \cite{herrmann_t_1994}. The notochord is induced by {\em bra} in both vertebrates and ascidians, with consistent timing and location expression of \textit{bra} in the notochord and mesoderm of mouse, xenopus, zebrafish and chicken \cite{kavka_tales_1997}, although {\em bra} is expressed exclusively in the notochord cells in ascidians \cite{yasuo_ascidian_1994}. Notochord induction is regulated by the \textit{FGF/MAPK/Ets} signaling cascade \cite{minokawa_binary_2001}. In particular, the A6.2 and A6.4 notochord/nerve cord precursors are induced by \textit{FGF9/16/20} at the 32-cell stage, just after the 7th cell cleavage \cite{satoh_ascidian_2001}. It was observed from isolation experiments that notochord/nerve cord precursors that lose \textit{FGF9/16/20} competence at the 32-cell stage assume the default nerve cord cell fate, but the converse is true for presumptive nerve cord blastomeres that are introduce to \textit{FGF}: they forgo their default nerve cord fate and become notochord \cite{yasuo_conservation_1998,minokawa_binary_2001}. If \textit{FGF9/16/20} is not present at the 32 cell stage competence is lost, \textit{bra} is not induced and the notochord no longer forms \cite{nakatani_basic_1996,nakatani_duration_1999}. This is because \textit{MAPK} is not activated and the induction of \textit{bra} and repression of \textit{FoxB} are not carried out \cite{hashimoto_transcription_2011}. Without the repression of \textit{FoxB TF} the notochord cell fate is repressed through the repression of \textit{bra}. It has been observed in \textit{H. roretzi} that \textit{FoxB} represses the activation of \textit{bra} predominately through the binding of Fox BS1 (GCACTGA\textit{ACAAACA}TACATAG). \textit{FoxB} is activated by \textit{ZicN} and is present in both nerve cord and notochords precursors, however it is repressed by \textit{MAPK} in the notochord cell lineage at the 64-cell stage \cite{hashimoto_transcription_2011}. \textit{MAPK} is thought to be repressed by \textit{Ephrin/Eph} signaling which is one of the key differences between notochord and nerve cord determination. \textit{Ephrin} ligand is expressed in the epidermis and signals to the future nerve chord cells, inhibiting \textit{FGF/MAPK} pathway. Notochord cells, not being in contact with the epidermis, are free to activate \textit{FGF/MAPK}, and activate \textit{bra}. At this point \textit{bra} is expressed first weakly in the at the 64-cell stage in the notochord/nerve chord precursors \cite{yasuo_ascidian_1994} and unlike other chordates, in ascidians \textit{bra} is only expressed in the notochord cells \cite{yasuo_function_1993,corbo_characterization_1997,hotta_temporal_1999,takada_brachyury_2002}. Although \textit{bra} is necessary, its presence does not guarantee a tail. \textit{M. occulta} and \textit{M. tectiformis}, two tailless \textit{Molgula}, both express \textit{bra}. In both cases \textit{bra} expression stops earlier than that of \textit{M. oculata}, while notochord development is slightly different between the two species. \textit{Bra} is expressed in the 10 precursor notochord cells in \textit{M. occulta}, which undergo another round of cell division, while this final division does not occur in \textit{M. tectiformis}.  In  both \textit{M. occulta} and \textit{M. tectiformis}, larva-specific muscle actin genes have become pseudogenes, however the mutation in the muscle actin genes are not the same between the two species  \cite{swalla_novel_1993,jeffery_evolution_1999}. \textit{Manx} is another gene identified to be important for tail development in \textit{Molgula}, and while it is lowly expressed in \textit{M. occulta}, it has been shown to restore the hybrid tail \cite{swalla_requirement_1996, swalla_multigene_1999}. 
  
After cell specification, the notochord cells must converge, intercalate and extend. The Planar Cell Polarity (PCP) pathway is involved in cell movement during this process and mutations in \textit{prickle}\textemdash a known PCP gene\textemdash have been shown to cause a shortened ascidian tail by affecting both the mediolateral intercalation and the elongation of the ascidian tail \cite{jiang_ascidian_2005}. The \textit{pk} mutant \textit{aimless} produces a truncated tail, however the polarity of the nuclei is established, showing that prickle does not establish polarity within the cell but polarity between cells, acting in a local manner and perhaps as a global organizer \cite{jiang_ascidian_2005,kourakis_one-dimensional_2014}. However, even in the absence of the PCP pathway considerable convergence and elongation of the notochord was observed in Ciona, driven by a presumed boundary effect \cite{veeman_chongmague_2008}.

Many of the upstream genes and transcription factors that interact with \textit{bra} have been studied in detail, through knock-outs and cell isolation experiments. A larger scale subtractive screening was done to identify genes downstream of \textit{bra}, in which 39 genes were initially found \cite{hotta_temporal_1999}. A number of these genes have been characterized, identifying functions such as extracellular matrix components (\textit{cadherin 8, entactin, fibronectin, laminin $alpha$1, $alpha$4, and $beta$1}, and {\em thrombospondin}), cell shape and polarity (\textit{pk, trop, ERM, ACL}), and axon guidance (\textit{netrin, semaphorin 3A}), amongst a host of other biological processes \cite{hotta_characterization_2000,hotta_brachyury-downstream_2007,kugler_evolutionary_2008}. Additionally, downstream genes regulated by \textit{bra} have been examined by using ChIP-seq to identify many known genes in the network, as well as to discover new genes \cite{kubo_genomic_2010,katikala_functional_2013}.

Larvaceans are pelagic tunicates that also develop in a typical chordate manner, featuring  a notochord. However, larvaceans notochords contain only 20 cells \cite{seo_miniature_2001,denoeud_plasticity_2010}. The larvacean  \textit{Oikopleura dioica} retains its tail during its adult life stage and at this point \textit{bra} is not expressed in the adult notochord, however, \textit{bra} is expressed in the same manner in the developing larval notochord as ascidians \cite{bassham_brachyury_2000,nishida_development_2008}. When comparing gene networks for the extent of variation, \textit{Oikopleura} did not exhibit the same mechanism for tail development as \textit{Ciona}: of the 50 {\em bra} target genes previously identified in \textit{Ciona}, only 26 of them had orthologs in the \textit{Oikopleura} genome, meaning that almost 50\% of candidate bra target  genes are not present \cite{kugler_evolutionary_2011}. Of the genes that did show homology, expression ranged from notochord-specific to tail-general, including expression in possible notochord to tissues that were clearly not the notochord. From this we can infer that additional genes have gained function in notochord formation in the \textit{C. intestinalis} gene regulatory network, and the ancestral tunicate may have had a small core set of genes for notochord formation.

\section{Assembling and analyzing data}

One of the major advances in biological sciences in the past 20 years has been the implementation of sequencing technologies. These technologies allow us to examine biological systems genomically, with increasing ease. The first widespread sequencing technology was Sanger sequencing in the 1986, but Sanger sequencing was not broadly used until 10 years later, when automated sequencers became available. Another technology, microarrays, which became popular starting in the mid '90s, allowed us to look at a wide spectrum of genes and understand relative expression within a sample. For example, Kobayashi et al. \cite{kobayashi_differential_2013} isolated and analyzed gene expression in notochord (A7.3+A7.7) and nerve cord (A7.4+A7.8) precursors using microarrays. This study was able to identify 106 genes expressed in the notochord precursor and 68 expressed in the nerve cord precursor at the 64-cell stage. Of these the genes, 36 notochord genes and 25 nerve chord genes were confirmed via whole mount in situ hybridization in the respective cells. This demonstrates the power of this technique, however, prior knowledge of the gene sequences involved were needed. \textit{C. intestinalis} was sequenced using Sanger sequencing, and is the best assembled and most well annotated ascidian genome \cite{dehal_draft_2002}. In addition to long (Sanger) reads, scaffolding was done using scaffold-joining guided by paired-end expressed sequence tags and bacterial artificial chromosome (BAC) sequences, and BAC chromosomal in situ hybridization data \cite{satou_improved_2008}. Sanger sequencing is able to sequence whole genomes without the need of prior knowledge to identify novel genes but was costly and time consuming\cite{metzker_emerging_2005,liu_comparison_2012}. 

Sanger was the first generation of sequencing technologies, and currently both second and third generation are in use, with Roche 454, Ion Torrent, Illumina and PacBio being the most widespread. These technologies produce data far more easily and at a much lower cost than Sanger sequencing \cite{metzker_emerging_2005}. There are many trade-offs for each of the technologies, including cost per MB, sequencing time, prep cost, error rate and sequencing bias; in particular, 454 and PacBio have longer reads than Illumina and Ion Torrent, 800 bp and 1+kbp, respectively. However, both Illumina and Ion Torrent's short reads are cheaper to generate, produce more reads and are better for digital counting; in addition, PacBio has a very high error rate \cite{glenn_field_2011}. Illumina and Ion Torrent have the lowest error rates and while Ion Torrent is more sensitive to single nucleotide polymorphisms, it also calls many more false positives.  Illumina has become the most used NGS technology because it is the most versatile and performs the best in general \cite{quail_tale_2012}. The associated drop in sequencing price has yielded many of the assembled genomes within the Tunicata phyla. Outside of this project there are ten tunicate genomes assembled; \textit{C. intestinalis} (species ``Type A'' and ``Type B''), \textit{C. savignyi}, \textit{Oikopleura dioica}, \textit{Botryllus schlosseri}, \textit{Halocynthia aurantium}, \textit{H. roretzi}, \textit{Phallusia fumigata}, \textit{P. mammilata}, and \textit{Didemnum vexillum}, but no \textit{Molgula} genomes. Molgula genomes and transcriptomes, specifically the tailed \textit{M. oculata} and the tail-less \textit{M. occulta} would make great systems to study the development of the chordate tail and chordate body plan. 
