\chapter{Tail loss?}

\section{Introduction}
Ascidians are an interesting system to study because of the relationship to the vertebrates. They form a tailed larvae with a hollow dorsal notochord before undergoing the process of metamorphosis. Of the \mytilde3000 species of ascidians less than 20 have been identified as undergoing tail-loss. Although each case of tail loss has happened independent of each other, many of the species tend to fall in closely related clades and many of them are Molgula. M. occulta, M. blezi . Although the mechanism behind tail-loss differs by species, a common characteristic is the lack of a notochord that intercalates and extends. M. bliezi notochord cells converge to the midline, and began to extend, however, cells never properly intercalated and the tail formation stop before it is fully formed. One reason behind this is the early down-regulation of \textit{bra} and another is the muscle actin becoming pseudo genes. It is thought that the early form of the notochord where not cartilage based but were made of muscle.  

----
With advances in high throughput sequencing technologies, gene expression of M. occulta, M. oculata, and hybrid species can be analyzed (Gyoja et al. 2007; Pickrell et al. 2010). The transcriptomes of three different developmental stages of M. occulta, M. oculata, and hybrids have been sequenced at Michigan State University. The three transcriptomes were used to identify the presence or absence of known notochord genes downstream of bra using C. intestinalis data from the NCBI database. BLAST searches were with known notochord genes, and several of them were selected for further analysis. FGF9/16/20, prickle (pk), and several other downstream brachyury factors?noto6, leprecan, merlin, and noto17?were analyzed for presence, temporal and spatial expression using in situ hybridizations. In addition to focusing on the notochord genes an EdgeR differential expression analysis was done to identify other genes that are involved in tail development.
----
\section{Methods}
\subsection{Sample collection, sequencing and assembly}
DNA was extracted from the gonads of an individual adult specimen for M. occidentalis, M. occulta, and M. oculata. Paired in jumping libraries were collected for each sample ranging from \mytilde300bp to \mytilde950bp. further details about extraction methods and libraries can be found in Stolfi et al., \cite{}. RNA was extracted from all three molgula species using the methods discussed in Lowe et al., \cite{}. Sequencing for M. occulta and M. oculata was conducted at the Michigan State University 

Genome assembly was conducted using 3-pass digital normalization\cite{} and assembled using Velvet\cite{}. Assemblies were done with 21 
Both de novo and reference based assembly were used when creating gene models. 
Notochord genes have been identified using subtractive hybridization, mircoarrays and 

\section{Results}
\subsection{\textit{M. occulta} and \textit{M. oculata} have strong overlap in gene presence}
\textit{C. intestinalis} is the closest ascidian species with a well annotated gene, because of these reason it was used to annotate the genomes of both \textit{M. occulta} and \textit{M. oculata}. 

-----
2.4.1 Preliminary results: Notochord genes

Reciprocal best hit (RBH) blast with an e-value of 1e-6, were done with the M. occulta and M. oculata transcriptomes against C. intestinalis for the annotation of each species. M. occulta and M. oculata have a high overlap in number of translated transcripts that showed homology with C. intestinalis proteins from the NCBI database. Of the 16 thousand proteins found in the NCBI database, both molgulid species recovered  ~84\% of the proteins. M. occulta had an additional 202 transcripts that were not found in M. oculata and M. oculata had an additional 250 transcripts that did not have hits in M. occulta, overlapping by 97\%. Next, we examined genes associated with notochord development in C. intestinalis to better analyze the molecular development of the tail.  Seventy-two genes identified as being involved in notochord development (Kugler et al 2008; Hotta et al 2000; Kugler et al. 2011; Jose?-Edwards 2011) were BLAST against each species. Four genes?not1, not4, not5, and snail?were not found in either of the species. This could have been due to true absence or sequence divergence in the transcripts. The remaining 68 genes were shared by both species with the exception of col5a, which was missing from M. occulta.

2.4.2	Prelliminary results: EdgeR differential expression analysis by stage 

Gene expression at the gastrula stage does not show a great deal of fold-change between M. occulta, M. oculata and the hybrid. The majority of the genes had a fold-change of less than 5 fold, and most of the genes clustered inside of that 5-fold window (Figure 3a).  The fact that the majority of the genes do not show significant fold-change is not surprising because M. occulta and M. oculata are very similar at this stage of development (Swalla and Jeffery 1990).  When comparing each species to the hybrid, most of the genes clustered within a 5-fold-change window. 
During neurulation M. occulta versus M. oculata differential expression pattern is similar to gastrula stage, clustering within a 5 fold-change window. However, there is a drastic different in the fact that ~300 of the genes have a 10 fold-change in expression (Figure 3d).  The change in expression between the two species mirrors what was observed in morphological studies (Swalla and Jeffery 1990). M. occulta has normal urodele?tailed?development up to gastrula and begins to diverge at neurulation. Of the genes that were higher expressed in M. oculata compared to M. occulta, those same genes were higher expressed in hybrids versus M. occulta. Those genes do not show significant differential expression when comparing M. oculata to hybrid. There are ~20 transcripts that show a 10 fold-change increase in expression in hybrid compared to M. oculata.
