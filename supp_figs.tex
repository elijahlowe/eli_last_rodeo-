\chapter{Supplemental Figures}
%\section{k}
%\setcounter{secnumdepth}{-1}
%\thispagestyle{plain}
%\renewcommand{\thefigure}{A.\arabic{figure}}
%\setcounter{figure}{0}
\begin{figure}[thbp]
\centering
\includegraphics[scale=0.6]{figures/Hox12_13.pdf}
\caption{\textbf{Alignment for \textit{hox12-13} in \textit{M. occulta}, \textit{M. oculata} and \textit{M. occidentalis}} The contig containing \textit{hox12-13} for \textit{M. occidentalis} and \textit{M. oculata}, along with the two contigs containing \textit{hox12} and \textit{hox13} for \textit{M. occulta}. \textit{M. oculata} was used as the anchor sequence because it showed the most similarity between the three species. Outside of the coding regions and its flanking area, there is very little sequence similarity between the species, and \textit{M. occidentalis} exclusively shows similarity in coding regions. Grey arrows show the direction of the contig.}
\label{fig:hox12}
\end{figure}

\begin{figure}[thbp]
\centering
\includegraphics[scale=0.75]{figures/occi_hox2.pdf}
\caption{\textbf{Alignment of \textit{M. occidentalis} \textit{hox2} genes with \textit{Ciona} show  premature stop codon.} The \textit{M. occidentalis hox2} gene has a stop codon in the 3' region, inside of the 3/4 helix. \textit{hox2} knockdowns in \textit{Ciona} did not show any phenotypic difference, so the function of \textit{hox2} may not be important in \textit{M. occidentalis} }
\label{fig:occihox2}
\end{figure}

\begin{figure}[tbp]
\centering
\includegraphics[scale=0.75]{figures/hox_alignment.pdf}
\caption{\textbf{Alignment of \textit{hox} genes.} Alignments of the aa homeobox sequences from all the Molgula species, show that they group with their respective orthologs. All but one of the \textit{M. occidentalis} cluster properly, but \textit{M. occidentalis hox10} full homeobox sequence was not fully assembled, so this is possibly the reason for poor clustering. }
\label{fig:hox-alignments}
\end{figure}

\begin{figure}[tbp]
\centering
\includegraphics[scale=0.95]{figures/Occi_hox10.pdf}
\caption{\textbf{Alignment of \textit{M. occidentalis} duplicate \textit{hox10} genes} Two copies of \textit{hox10} were found in \textit{M. occidentalis} \mytilde12 kb apart on the same contig.}
\label{fig:occihox10}
\end{figure}

\begin{figure}[tbp]
\centering
\includegraphics[scale=0.10]{figures/gel.jpg}
\caption{\textbf{Gel electrophoresis of cdc45, netrin and controls} \textit{Netrin} was not found in \textit{M. occulta's} transcriptome but was recovered via PCR from a cDNA library. However, this library has a wider range of developmental stages, so it is still possible \textit{netrin} is not expressed at the right stage for tail development.}
\label{fig:gel}
\end{figure}

\begin{figure}[tbp]
\centering
\includegraphics[scale=0.55]{figures/prickle.pdf}
\caption{\textbf{Whole Mount In Situ Hybridization of \textit{prickle} in \textit{M. occulta}} The PCP gene \textit{pk} was shown to be expressed in the notochord in a similar pattern to \textit{C. intestinalis} }
\label{fig:prickle}
\end{figure}
