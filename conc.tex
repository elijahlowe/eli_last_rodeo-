\chapter{Conclusions}
The chordate body plan is conserved throughout the phyla with several species deviating; most chordates develop a tadpole larvae containing a hollow dorsal neural tube, and a postanal tail containing a notochord flanked by muscle cells at some point in their life cycle. %Our work gives insight into the mechanisms behind the chordate body plan. 
Molgulids are especially useful models for studying changes in the development of body plans because they have both tailed and tail-less species, two of which have the ability to produce interspecific hybrids \cite{swalla_interspecific_1990}. This shows us that larval development can evolve rapidly, and allows us to examine the mechanisms of evolutionary change at an allele-specific level.

\section{Evaluating a lightweight transcriptome assembly pipeline on two closely related ascidian species}
Many non-model systems are now being sequenced with the drop in sequencing price. The methodology for assembly is not cut and drier: there are a number of different steps\textemdash quality trimming, filtering, and choice of assembler(s), with several programs at each step and no clear choice. Studies have been done to test which assembler is best, in which it was determined no one assembler is the best \cite{clarke_comparative_2013}. Many times assembly methods are chosen on the usability of software and the availability of resources. Factors that are most limiting for assemblies are memory, and in some cases time. The redundant nature of sequencing data allows for the removal of redundant reads, but at what cost? Here in efforts to assemble the first two molgula transcriptomes, we evaluate the cost of redundancy filtered sequencing reads. We show that pre-filtering quality trimmed assembly reads does not reduce the information content of the assembly, but a reduction in computational resources. We have demonstrated that the Oases and Trinity assemblers return similar results, both suitable for downstream analysis using the full or redundancy filtered dataset. Our pipelines are available so we also provide methodology to be used by future researchers. In addition assembly metrics to evaluate assemblies are also an important step in the protocol. One of the standard metrics for evaluating assemblies is the N50, which has been designed for genomes and does not clearly translate to transcriptomes because of their fragmented nature, and the likelihood of chimeric contigs; homology and programs such as CEGMA are more informative and useful for downstream analysis. %When only taking N50 into account 
 
The two \textit{Molgula} species that were assembled are evolutionarily interesting because of the position within the chordates and \textit{M. occulta}'s divergent body plan. The downstream analysis of these two molgula tanscriptomes will give insight into the formation of the chordate body plan. 

\section{Genome assembly and characterization}
In tunicates \textit{hox} genes are not important for patterning along the anterior-posterior axis, as they are in vertebrate and other bilaterians \cite{finnerty_origins_2003,mallo_regulation_2013,ikuta_limited_2010}. This limited function of the \textit{hox} genes is evident, from the loss and rearrangement of several genes in the ascidian \textit{hox} clusters, in addition to no noticeable phenotypic defects in the knockdown of \textit{C. intestinalis hox} genes, with the exception of \textit{hox10} and \textit{hox12}. \textit{Hox10} is involved in neuronal development and \textit{hox12} is involved the formation for the posterior most portion of the the tunicate larval tail \cite{ikuta_limited_2010}. There have yet to be two tunicate species studied with the same \textit{hox} cluster configuration, perhaps after diverging from vertebrae and cephalochordates, the inconsistency in the function of the hox clusters, tunicates were loss in to conserve tunicates highly invariant  body plan and the function of the \textit{hox} genes  were co-opted by other genes. % leading the hox gene clusters to eventually 

Hox lost its function in tunicates after diverging from vertebra and cephalochordates but before the divergence of tunicates. 

We cannot be sure that dll is not present in the M. occulta genome because the genome is not complete, however  \textit{Molgula}

In addition we have identified a number of enhancer between \textit{C. intestinalis}, \textit{M. occidentalis}, and \textit{M. oculata} that were either mutually unintelligible, or asymmetrically intelligible. We propose that these changes are required for the rapidly evolving ascidians to accommodate the constraints imposed by their invariant embryonic cell lineages and highly compact genomes.

Better scaffolding to identify the arrangement of the hox clusters. 

identify less board regulation regions, and perhaps identify the causes for lose in across species regulation.

The sequencing of these organisms adds to the body of knowledge allowing ups to examine a wider range of chordates and the similarities and drast differences. 

\section{Differential expression analysis of tail loss in an invertebrate chordate}
Studying closely related organisms gives us insight into the underlying evolutionary mechanisms of divergence and underlying development. Indirect and direct developing sea urchins have been studied showing that change in axial cleavage patterns and early cell fate leads to divergent larval body plans that can exhibits similar adult body plans \cite{wray_evolutionary_1989,henry_evolutionary_1990}. In contrast, ascidians retain their typical cell division and invariant cell fate in both direct, indirect, tailed and tail-less ascidians \cite{jeffery_evolutionary_1991,maliska_molgula_2010}. Next generation sequencing is a great way to study non-model species. A broad swath of information can be gained from both RNA and DNA sequencing. 

This technology has allowed us to analyzed two closely related invertebrate chordates that also hybridize. This study has given us insight into the mechanisms behind tail-loss and development. Gene loss has occurred but other factors appear to be involved in tail-loss because expression from the tailed species appears to recover the loss features. 
  
We have shown that evolutionary drift has occurred in the cis-regulatory and are not always capable of driving expression in other ascidian species. It appears that the same has happened between the tail-less \textit{M. occulta}, however those cis mechanisms are restored in the (\textit{M. oculata x M. occulta}) interspecific hybrids. Further research is needed to identify and test cis elements, but as of now we have a better understanding of the mechanism behind different larval body plans in the Molgulids. 

Differential expression is a known cause for different phenotypes, but the mechanisms are not always understood. In this case change in the cis-regulatory elements appears to have lead to differential expression.

additional sequencing is need.

