\chapter{Conclusions and Discussion}
The chordate body plan is conserved throughout the phyla with several species deviating; most chordates develop a tadpole larvae containing a hollow dorsal neural tube, and a postanal tail containing a notochord flanked by muscle cells at some point in their life cycle. However, it has been documented that 16 species out of the approximately 3000 species of tunicates?which are one of the three subphyla of chordates?have independently undergone tail loss. With the majority (percentage) being within the moluglids.%Our work gives insight into the mechanisms behind the chordate body plan. 
Molgulids are especially useful models for studying changes in the development of body plans because they have both tailed and tail-less species, two of which have the ability to produce interspecific hybrids \cite{swalla_interspecific_1990}. Having closest related species with alternate developmental allows us to look at the immediate switches that occurred during the evolution of alternate body plans. This shows us that larval development can evolve rapidly, and allows us to examine the mechanisms of evolutionary change at an allele-specific level.

\section{Evaluating a lightweight transcriptome assembly pipeline on two closely related ascidian species}
The drop in sequencing price has aided in our efforts to understand body plan development. %Many non-model systems are now being sequenced with the drop in sequencing price. 
The methodology for assembly is not unambiguous : there are a number of different steps\textemdash quality trimming, filtering, and choice of assembler(s), with several programs at each step and no clear choice. Studies have been done to test which assembler is best, in which it was determined no one assembler is the best \cite{clarke_comparative_2013}. Many times assembly methods are chosen on the usability of software and the availability of resources. Factors that are most limiting for assemblies are memory, and in some cases time. The redundant nature of sequencing data allows for the removal of redundant reads, but at what cost? Here in efforts to assemble the first two molgula transcriptomes, we evaluate the cost of redundancy filtered sequencing reads. We show that abundance filtering quality trimmed assembly reads enables transcriptome assembly with a reduction in both memory requirement and assembly time, while retaining essentially the same information content (e.g. number of genes, and average gene length). We have demonstrated that the Oases and Trinity assemblers return similar results, both suitable for downstream analysis using the full or redundancy filtered dataset. Our pipelines are available so we also provide methodology to be used by future researchers. In addition assembly metrics to evaluate assemblies are also an important step in the protocol. One of the standard metrics for evaluating assemblies is the N50, which has been designed for genomes and does not clearly translate to transcriptomes because of their fragmented nature, and the likelihood of chimeric contigs; homology and programs such as CEGMA are more informative and useful for downstream analysis. %When only taking N50 into account 
 
The two \textit{Molgula} species that were assembled are evolutionarily interesting because of the position within the chordates and \textit{M. occulta}'s divergent body plan. The downstream analysis of these two molgula tanscriptomes will give insight into the formation of the chordate body plan. 

\section{3.	Change in gene function and cis-regulatory binding sites}
In tunicates \textit{hox} genes are not important for patterning along the anterior-posterior axis, as they are in vertebrate and other bilaterians \cite{finnerty_origins_2003,mallo_regulation_2013,ikuta_limited_2010}. This limited function in the hox genes is evident from no noticeable phenotypic defects in the knockdown of C. intestinalis hox genes, with the exception of hox10 and hox12. \textit{Hox10} is involved in neuronal development and \textit{hox12} is involved the formation for the posterior most portion of the the tunicate larval tail \cite{ikuta_limited_2010}. In addition, tunicates have been identified to undergo loss and rearrangement within the hox clusters. There have yet to be two tunicate species studied with the same \textit{hox} cluster configuration, there are changes in ordering, duplications, and which hox genes are present (figure 4.2). Only C. intestinalis has enough scaffolding to indentify the intergenic spacing between the hox genes. However, of the hox gene found on the same scaffolds in M. occ, M. ocu and M.occi the average intergenic spacing had far smaller intergenic regions range from 10-25 kb in length for the hox genes that were found on the same contigs. With out additionally scaffolding we cannot truly tell the level at which the rearrangement was occurred within the hox clusters of asidians. Perhaps after diverging from vertebrae and cephalochordates, the inconsistency in the function of the hox clusters was established in tunicates to conserve their highly invariant  body plan and the function of the \textit{hox} genes  were co-opted by other genes to establish a more robust developmental program. % leading the hox gene clusters to eventually 

In addition we have identified a number of enhancer between \textit{C. intestinalis}, \textit{M. occidentalis}, and \textit{M. oculata} that were either mutually unintelligible, or asymmetrically intelligible. 3.	We propose that a high frequency of compensatory changes are required to for the rapidly evolving ascidians to maintain their invariant embryonic cell lineages in the presence of their compact genomes. These compensatory changes have caused the enhancers to diverge within the ascidians while allowing them to develop with their typical body plan. However, for this same reason are more susceptible to undergoing a change of body plan when the high frequency of compensatory changes in sequence passes an allowable threshold.

\section{Differential expression analysis of tail loss in an invertebrate chordate}
Studying closely related organisms gives us insight into the underlying evolutionary mechanisms of divergence and underlying development. Indirect and direct developing sea urchins have been studied showing that change in axial cleavage patterns and early cell fate leads to divergent larval body plans that can exhibit similar adult body plans \cite{wray_evolutionary_1989,henry_evolutionary_1990}. In contrast, ascidians retain their typical cell division and invariant cell fate in both direct, indirect, tailed and tail-less ascidians \cite{jeffery_evolutionary_1991,maliska_molgula_2010}. %Next generation sequencing is a great way to study non-model species. A broad swath of information can be gained from both RNA and DNA sequencing. 
   
We have shown that evolutionary drift has occurred in the cis-regulatory modules of developing ascidian embryos between C. intestinalis and M. occidentalis, and these enhancers are not always capable of driving expression in other ascidian species. It appears that the same has happened between the tailed and tail-less \textit{M. occulta}, however those cis mechanisms are restored in the (\textit{M. oculata x M. occulta}) interspecific hybrids, allowing TF to bind their targets at a higher affinity, thus restoring the necessary level of expression to develop the urodele features.. Further research is needed to identify and test cis elements, but as of now we have a better perception behind the mechanism the restoration of the larval tail, which appears to be driven by CRM.

\section{Conclusion}

When examining the tailed and tail-less Molgula?M. oculata and M. occulta?there is a strong overlap (91\%) in the expression of genes that can be annotated using the C. intestinalis proteom. There was also a strong overlap in genes associated with notochord development, which is the key structure-defining element not present in M. occulta. We observed it was not the lack of genes in the transcriptome, but the expression of said genes, which has led to the tail-less phenotype. So, the identification of regulatory gene expressed differently in the urodele and anural species is needed to understand the molecular mechanism underlying the evolutionary transition from urodele to anural development. Here we were able to see in the hybrids that the restored expression came from the tailed allele and we proposed that cis-regulatory modules are the cause of this. The tailed allele restored the enhancer binding sites and TF were able to bind at a higher affinity and restore necessary level of expression to generate urodele features, including the larval tail through the formation of the notochord.

One of the shortcomings of our study is the lack of replicates from all stages, ideally we would like at three per developmental stage for each organism. We used the available replicates to estimate the dispersion for each of the samples and because of this there are sure to be more false positives and false negatives in our analysis in comparison to having the appropriate number of replicates for each sample. However, examining the correlation between two independent datasets, the tailed M. oculata and the hybrid, we were able to identify a subset of genes that we believe to be true (better word than truly) differentially expressed genes. Replicates are needed, and addition sequencing of other tailed and tail-less molgula genomes and transcriptomes, which would give us a better insight into why the molgula are able to easily undergo tail-loss and the toolkit for molgula tail development. 

Our hypothesis depends heavily on the assumption that the gene regulatory network (GRN) is conserved within the tunicates to understand the mechanisms behind tail loss, and the GRN is conserved throughout the chordates to understand the origin the chordate body plan. Comparing the GRN of distantly related organisms the necessary genes involved in the development of similar phenotypes and body plans. However, it is necessary to determine if the GRN work in the same fashion, first by confirming the spatial expression patterns for the network. To identify TF we will first annotate the genes with gene ontology (GO) terms and select differently expression genes associated with DNA binding. Once we indentify our candidate TF we will process with WMISH. The GO analysis and WMISH analysis what process the genes are associated with and in what cells. 

As stated above, the fact that the paternal genome restores the urodele features leads us to believe that the presence CRM and TF binding affinity are a major cause of tail loss, this is supported by our analysis; in hybrids the expression of up-regulated genes involved in development of urodele features are predominately from the paternal allele. To confirm this hypothesis we would first perform a ChIP-sequence analysis to identify direct binding targets, and correlate them with the DE genes. From this analysis we can examine the divergence between M. occulta and M. oculata enhancers to test their binding affinities, and examine the found sites using transgenesis to see if we can restore the M. occulta's tail.

=====================
Identification of regulatory gene expressed differently in the urodele and anural species is needed to understand the molecular mechanism underlying the evolutionary transition from urodele to anural development. We identifed genes that may be involved in this developmental body plan transition from tailed to tail-less, however, without replicates we cannot call differential expressed genes with statistical confidence, and there are sure to be both false positives and false negatives in our analysis. However, examining the correlation between the tailed \textit{M. oculata} and the hybrid, we were able to identify a subset of genes that we believe to be true differentially expressed genes. Replicates are needed, and addition sequencing of other tailed and tail-less \textit{molgula} genomes and transcripotmes would give us a better insight into why the \textit{molgula} are able to easily undergo tail loss and the toolkit for \textit{molgula} tail development. The fact that the zygotic genome restores the urodele features leads us to believe that cis-acting factors are a major cause of tail loss, this is supported by our analysis; in hybrids the expression of up-regulated genes involved in development of urodele features are predominately from the zygotic allele.

Our hypothesis depends heavily on the assumption that the gene regulatory network (GRN) is conserved within the tunicates to understand the mechanisms behind tail loss, and the GRN is conserved throughout the chordates to understand the origin the chordate body plan. Currently we plan to expand our analysis by comparing the \textit{molgula} to the ten tunicate genomes assemblies found on the aniseed website (\url{http://www.aniseed.cnrs.fr/}), and available chordate genomes found on NCBI (\url{http://www.ncbi.nlm.nih.gov/}) and ensembl (\url{ensembl.org}). With the information we currently have we will give deeper look into the sequence divergence of both gene and non coding regions. To examine the noncoding regions addition genomic sequencing is needed for gap filling and scaffolding. Furthermore, molecularly, we will examine the spatial expression through whole mount in situ hybridization to confirm orthologous function.  Additionally we would like to identify binding sites using Chip-seq and examine the found sites using transgenesis to see if we can restore the \textit{M. occulta's} tail.
